\documentclass{article}% insert '[draft]' option to show overfull boxes 

%set up margins (gemetry package does more than just margins.)
\usepackage{geometry}
\geometry{
	letterpaper,
	left=1in,
	top=1in,
	bottom=1in}

%% Useful packages
%Nominal Packages
\usepackage{amsmath,xparse} %package for math stuff
\usepackage{esint} %package for cauchy principal value integral \fint
\NewDocumentCommand{\qfrac}{smm}{%
	\dfrac{\IfBooleanT{#1}{\vphantom{\big|}}#2}{\mathstrut #3}%
} % spaces out fractions

\usepackage{graphicx} %package for floats (figures)
\usepackage[colorlinks=true, allcolors=blue]{hyperref} %pacakge for hyperlinks (internal and external)
\usepackage{cleveref}
\usepackage{indentfirst} %package that indents first sentence of each paragraph.

%Advanced Packages
\usepackage{xfrac} %package that allows slanted fractions rather than stacked ones, use \xfrac{}{} instead of \frac{}{}
\usepackage{siunitx} %package for SI units (see ftp://ftp.dante.de/tex-archive/macros/latex/exptl/siunitx/siunitx.pdf)
\usepackage{gensymb} %package for some convenient things in both math and text mode (see http://ctan.math.illinois.edu/macros/latex/contrib/was/gensymb.pdf)
\usepackage{caption} %package for advanced captioning
\usepackage{subfigmat}% packages automating layout for subfigures
\usepackage{tcolorbox} %makes colored boxes

\usepackage{listings}
%\usepackage[usenames,dvipsnames]{xcolor}

\lstdefinelanguage{Julia}%
{morekeywords={abstract,break,case,catch,const,continue,do,else,elseif,%
		end,export,false,for,function,immutable,import,importall,if,in,%
		macro,module,otherwise,quote,return,switch,true,try,type,typealias,%
		using,while},%
	sensitive=true,%
	alsoother={\$},%
	morecomment=[l]\#,%
	morecomment=[n]{\#=}{=\#},%
	morestring=[s]{"}{"},%
	morestring=[m]{'}{'},%
}[keywords,comments,strings]%

\lstset{%
	language         = Julia,
	basicstyle       = \footnotesize,
	keywordstyle     = \bfseries\color{blue},
	stringstyle      = \color{magenta},
	commentstyle     = \color{gray},
	showstringspaces = false,
	breaklines=true,
	breakindent=0pt,
	tabsize=4,
}


\usepackage{fancyhdr}
\pagestyle{fancy}
\fancyhf{}
\rhead{Last Updated\\\today}
\chead{Activity 4:\\Plotting}
\lhead{FLOW Lab\\LTRAD Program}
\rfoot{}
\lfoot{}


%Document Actually begins here.
\begin{document}

The purpose of this assignment is to gain familiarity with basic plotting. You can learn about these with the \href{https://en.wikibooks.org/wiki/Introducing_Julia/Plotting}{Introducing Julia} page.

\bigskip

You won't need any more functions for this activity. In order to practice plotting, do the following:
\begin{enumerate}
	\item Using your NACA 4-series coordinate function from activity 3, plot a NACA 2412 airfoil.
	\item Explore the 3 NACA parameters by plotting (and labeling) several NACA airfoils (you may try a 0008, 0012, 2412, 4412, 2424, etc.)
	\item Adjust several line parameters such as line color, style, and thickness.
	\item Learn how to turn the grid off/on.
	\item Adjust the legend to have no border and a transparent background.
	\item Learn how to adjust the x and y limits of the plot, as well as the x and y tick labels.
	\item Save your figure as a .pdf file. 
\end{enumerate}

When you are done, you should have a figure resembling \cref{fig:airfoilplot}


\begin{figure}[h!]
	\centering
	\includegraphics[width=\textwidth]{activity4plot.pdf}
	\caption{An example output for this activity.}
	\label{fig:airfoilplot}
\end{figure}


\end{document}