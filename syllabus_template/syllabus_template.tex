\documentclass[12pt]{article}
\usepackage[top=1.0in,
bottom=1.0in,
right=1.0in,
left = 1.0in]{geometry}
\usepackage{graphicx}
\usepackage{datetime}
\usepackage[hidelinks]{hyperref}
\usepackage[noabbrev]{cleveref}
\usepackage{multicol}

\def\labelitemi{--}

\usepackage{fancyhdr}
\pagestyle{fancy}

\fancyhead[L]{ME EN 497R Syllabus}
\fancyhead[C]{Semester YYYY}
\fancyhead[R]{Student Name}
\fancyfoot[L,C]{}
\fancyfoot[R]{\thepage}



\begin{document}
	
\section*{Overview}
%NOTE: This section has been written for the student's information.  Make sure they see it before you overwrite it with the overview for their semester.
ME EN 497R is a research course that may be applied toward the technical elective requirement for the Mechanical Engineering program (and occasionally other programs, e.g. Applied Physics).
The overall goal of the 497R course is to give undergraduates research experiences typically reserved for graduate students, albeit in an format accessible to the abilities and time constraints of an undergraduate student.
In the FLOW Lab, under the direction of Dr. Ning, we have somewhat formalized this course while also allowing a great deal of student-lead learning.
This syllabus document is the beginning of that process, wherein the undergraduate and graduate mentor work together to develop a customized syllabus for the undergraduate.
This document is a template to be filled out and submitted to Dr. Ning for final approval and an add code for ME EN 497R.
This paragraph should be replaced by an overview paragraph summarizing the rest of this document.
Note that items in square brackets below should be replaced with the relevant information based on each student's custom course design.
Also note, that there are several auxiliary items that every student will be expected to accomplish during their 497R experience.  These are the items not within square brackets.

\section*{Learning Outcomes}

\subsection*{Technical}


\begin{enumerate}
	\item[1.] [Technical Outcomes Here]
	\begin{itemize}
		\item[--] [futher details as needed]
	\end{itemize}
\end{enumerate}

\subsection*{Auxiliary}
%NOTE: DO NOT CHANGE THIS SUBSECTION!!!

\begin{enumerate}
	\item Coding: The student, utilizing the Julia language, will learn the follwing:
	\begin{itemize}
		\item Docstring usage
		\item Good commenting practice
	\end{itemize}
	\item Version Control: Using git, the student will gain familiarity with the following concepts:
	\begin{itemize}
		\item Using github Projects
		\item Submitting and responding to github Issues
		\item Opening and reviewing github Pull Requests
	\end{itemize}
	\item Technical Writing: The student will gain introductory experience in the following areas:
	\begin{itemize}
		\item Reading and reviewing technical papers
		\item Writing technical memos, a research proposal, and a technical report
		\item Presenting technical information in well designed figures
		\item Receiving and applying critical feedback
	\end{itemize}
\end{enumerate}

\section*{Semester Schedule}

As mentioned above, the semester will be broken down into 8, 2-week mini-projects as follows:

\subsection*{Weeks 1-2: [Assignment 1]}

\subsection*{Weeks 3-4: [Assignment 2]}

\subsection*{Weeks 5-6: [Assignment 3]}

\subsection*{Weeks 7-8: Prospectus}

Each student will be required to submit a brief research prospectus including a simple literature review (at least 3 sources).  See the prospectus in the undergraduate-onboarding repository template for additional information.

\subsection*{Weeks 9-10: [Assignment 4]}

\subsection*{Weeks 11-12: [Assignment 5]}

\subsection*{Weeks 13-14: [Assignment 6]}

\subsection*{Weeks 15-16: Final Report and Code Submission}

The Final Report is a culmination of the bi-weekly memos produced thus far, revised according to mentor feedback, as well as any additional material necessary to produce a technical report that is both complete and well executed.  A final report template is available in the undergraduate-onboarding repository

The Final Code Submission is similarly the final, cumulative state of the code produced as part of the bi-weekly assignments, revised according to mentor feedback, as well as any additional material necessary to understand and use the code produced.

Both submissions will be due to the graduate mentor for review/grading on the last day of Finals.

\section*{Grade Rubric}

\Cref{tab:rubric} includes the proprosed grading rubric for this course.

\begin{table}[h!]
	\caption{Proposed grade division for ME 497R. }
	\label{tab:rubric}
	\renewcommand{\arraystretch}{1.2}
	\vspace*{1em}
	\begin{tabular}{c|l}
		\textbf{Points} & \textbf{Deliverable}\\ 
		\hline
		10 & [Assignment 1] \\
		10 & [Assignment 2] \\
		10 & [Assignment 3] \\
		10 & Prospectus     \\
		10 & [Assignment 5] \\
		10 & [Assignment 6] \\
		10 & [Assignment 7] \\
		10 & Final Report   \\
		80 & Time Log       \\
		   &                \\
		\textbf{160} & \textbf{Total} \\
		
	\end{tabular}
\end{table}

\newpage

\section*{Student/Mentor Acknowledgement}

I, [the student], have helped write, and have read, and I understand the above proposed syllabus and accept it as my desired 497R experience.

\vspace{2em}

\noindent \makebox[2.5in]{\hrulefill} \hspace {1.0in}\makebox[2.5in]{\hrulefill} \\
Student Signature \makebox[2.5in][r]{ Date} \\

\bigskip

I, [the mentor], have read the above proposed syllabus and accept resposibility for providing guidance, feedback, and grades as required by this course plan.

\vspace{2em}

\noindent \makebox[2.5in]{\hrulefill} \hspace {1.0in}\makebox[2.5in]{\hrulefill} \\
Mentor Signature \makebox[2.5in][r]{ Date} \\

\end{document}