%%%%%%%%%%%%%%%%%%%%%%%%%%%%
%%%%%     PREAMBLE     %%%%%
%%%%%%%%%%%%%%%%%%%%%%%%%%%%

\documentclass[12pt]{article}

%Define Margins using Geometry Package
\usepackage[top=1.0in,
bottom=1.0in,
right=1.0in,
left = 1.0in]{geometry}


%math
\usepackage{mathtools}


%Set up links and internal references
\usepackage[colorlinks=true,linkcolor=blue, urlcolor=blue]{hyperref} %hide the ugly red boxes
\usepackage[noabbrev]{cleveref} %don't abbreviate ``figure'', etc.
\usepackage{url} %for urls

%Remove natural indentation for itemize and enumerate environments
\usepackage{enumitem} %package to control itemize/enumerate behavior
\setlist[itemize]{leftmargin=*} %for itemize
\setlist[enumerate]{leftmargin=*} %for enumerate
%Swap out the itemize symbol for an N-dash rather than a bullet (does not require enumitem package)
\def\labelitemi{--}


%Set up fancy header and footer
\usepackage{fancyhdr}
\pagestyle{fancy}
%Fancy Header Content
\fancyhead[L]{ME EN 497R}
\fancyhead[C]{}
\fancyhead[R]{Airfoil Family Assignment A}
%Fancy Footer Content
\fancyfoot[L,C]{} %make bottom left and center empty
\fancyfoot[R]{\thepage}

%Center section headings
\usepackage{sectsty}
\sectionfont{\centering}
%Remove section numbering
\setcounter{secnumdepth}{0}




%%%%%%%%%%%%%%%%%%%%%%%%%%%%%%%%
%%%%%     END PREAMBLE     %%%%%
%%%%%%%%%%%%%%%%%%%%%%%%%%%%%%%%


%BEGIN Document Environment
\begin{document}
	
	\section{Background}
	
	Airfoils are the cross sections of wings and rotors, so airfoil performance directly affects the performance of any lifting object.
	You can learn more about the theory from the ME 515 Book and/or a google search.
	
	For this assignment, you will be utilizing a code produced by students in the FLOWLab.
	It is a Julia package called Xfoil.jl.
	You will probably want to go through the examples in the documentation to get familiar with how to use the code.
	
	[todo: add more details to the background]
	
	\section{Assignment}
	Once you are familiar with using Xfoil.jl, complete the following:
	
	\begin{enumerate}
		\item Explore the effect of airfoil thickness on airfoil lift and drag. 
		\item Explore the effect of airfoil camber on airfoil lift, drag, and moment.
		\item Explore the effect of airfoil angle of attack on airfoil lift, drag, and moment.
		\item Generate a surrogate model for a family of airfoils of your choice/creation.
	\end{enumerate}
	[todo: nail down what you actually want to be accomplished in this assignment.  Looks like 2 different assignments right now.]
	
	\noindent Then write a report (paper) on your methods, results, and takeaways as described in the course syllabus.
	
	\bigskip
	
	\noindent And submit your code via a branch/pull request as described in the course syllabus.
	
	\section{Useful Resources}
	
	[todo: add links here to text resources, code documentation, etc.]
	
\end{document}