%%%%%%%%%%%%%%%%%%%%%%%%%%%%
%%%%%     PREAMBLE     %%%%%
%%%%%%%%%%%%%%%%%%%%%%%%%%%%

\documentclass[12pt]{article}

%Define Margins using Geometry Package
\usepackage[top=1.0in,
bottom=1.0in,
right=1.0in,
left = 1.0in]{geometry}


%math
\usepackage{mathtools}


%Set up links and internal references
\usepackage[colorlinks=true,linkcolor=blue, urlcolor=blue]{hyperref} %hide the ugly red boxes
\usepackage[noabbrev]{cleveref} %don't abbreviate ``figure'', etc.
\usepackage{url} %for urls

%Remove natural indentation for itemize and enumerate environments
\usepackage{enumitem} %package to control itemize/enumerate behavior
\setlist[itemize]{leftmargin=*} %for itemize
\setlist[enumerate]{leftmargin=*} %for enumerate
%Swap out the itemize symbol for an N-dash rather than a bullet (does not require enumitem package)
\def\labelitemi{--}

\usepackage{multicol} %Allow for two columns in the middle of a paper. 


%Set up fancy header and footer
\usepackage{fancyhdr}
\pagestyle{fancy}
%Fancy Header Content
\fancyhead[L]{ME EN 497R}
\fancyhead[C]{}
\fancyhead[R]{Rotor Design - Panel Method Assignment}
%Fancy Footer Content
\fancyfoot[L,C]{} %make bottom left and center empty
\fancyfoot[R]{\thepage}

%Center section headings
\usepackage{sectsty}
\sectionfont{\centering}
%Remove section numbering
\setcounter{secnumdepth}{0}




%%%%%%%%%%%%%%%%%%%%%%%%%%%%%%%%
%%%%%     END PREAMBLE     %%%%%
%%%%%%%%%%%%%%%%%%%%%%%%%%%%%%%%


%BEGIN Document Environment
\begin{document}
	
	\section{Rubric}
	
	Here is the breakdown for the points.  Note that the breakdown follows a tree diagram, so any sub-bullet is the breakdown of the parent bullet's point. For instance, the entire assignment is worth 60 points. The exploration part is worth 35 of the 60 points (58\% of the total score). Writing skill is worth 6 of those 35 points. And so on.  
	
	\bigskip
	
	\textbf{Total Assignment - 60 points}
	
\begin{itemize}
	\item (35 points) Exploration Report
	\begin{itemize}
		\item (6 points) Writing Skills
		\begin{itemize}
			\item (2 points) Correct grammar usage.
			\item (1 point) Includes citations.
			\item (1 point) Includes figures.
			\item (1 point) Is in the IMRAD format (organized well).
			\item (1 point) The report looks pretty.
		\end{itemize}
		\item (5 points) Appendix Dictionary
		\begin{itemize}
			\item (1 point) Dictionary is included.
			\item (1 point) Dictionary is well though out. Effort is clearly demonstrated.
			\item (1 point) Figures are included.
			\item (1 point) Equations are included.
			\item (1 point) Recommended words are included.
		\end{itemize}
		\item (9 points) Explores effect of angle of attack on lift, drag, and moment.
		\begin{itemize}
			\item (6 points) Discusses major characteristics of airfoil polar.
			\begin{itemize}
				\item (2 points) Talks about lift, drag, and moment
				\item (1 points) Talks about how lift, drag, and moment are all just functions of alpha... $C_L(\alpha)$, $C_D(\alpha)$, and $C_M(\alpha)$. ... All of these values are inter-related to the angle of attack of the airfoil. 
				\item (3 points) Talks about major trends, such as stall, zero-lift angle of attack. 
			\end{itemize}
			\item (3 points) Includes a figure for lift, drag, and moment (1 point each)
		\end{itemize}
		\item (4 points) Explores effect of airfoil thickness
		\begin{itemize}
			\item (1 point) Includes a relevant figure. 
			\item (1 point) Explains thickness
			\item (2 points) Discusses the relationship between thickness and lift (drag and moment optional). 
		\end{itemize}
		\item (4 points) Explores effect of airfoil camber
		\begin{itemize}
			\item (1 point) Includes a relevant figure. 
			\item (1 point) Explains camber
			\item (2 points) Discusses the relationship between camber and lift (drag and moment optional). 
		\end{itemize}
		\item (3 points) Airfoil Polar Comparison
		\begin{itemize}
			\item (1 point) Includes a relevant figure. 
			\item (1 point) Compares student's calculated values and experimental/accepted values.
			\item (1 point) Includes error calculations.
		\end{itemize}
		\item (4 points) Explores effect of Reynold's Number
		\begin{itemize}
			\item (1 point) Includes a relevant figure.
			\item (1 point) Talks about the Reynold's number.
			\item (2 point) Discusses the relationship between the Reynold's number and lift (drag and moment optional).
		\end{itemize}
	\end{itemize}
	\item (25 points) Research Skills
	\begin{itemize}
	\item (5 points) Template preamble comments.
	\begin{itemize}
		\item (1 point) Preamble comments are present.
		\item (3 points) Comments are correct.
		\item (1 point) Comments are pretty and concise.
	\end{itemize}
	\item (5 points) Used a branch.
	\item (5 points) Used issues properly.
		\begin{itemize}
		\item (2 points) Issues were used.
		\item (2 points) Issues were closed with a comment.
		\item (1 point) Issues showed thought and effort.
	\end{itemize}
	\item (5 points) Used a pull request.
		\begin{itemize}
		\item (3 points) Submitted assignment via a pull request.
		\item (2 points) Pull request comment demonstrated thought. 
	\end{itemize}
	\item (5 points) Code was submitted.
		\begin{itemize}
		\item (4 points) Code used was submitted.
		\item (1 point) Code was of good style (not super messy). 
	\end{itemize}
	\end{itemize}
\end{itemize}



	
	

	
	
	
	
	
	
\end{document}