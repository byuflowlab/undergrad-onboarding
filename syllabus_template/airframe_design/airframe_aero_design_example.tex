%%%%%%%%%%%%%%%%%%%%%%%%%%%%
%%%%%     PREAMBLE     %%%%%
%%%%%%%%%%%%%%%%%%%%%%%%%%%%

\documentclass[12pt]{article}

%Define Margins using Geometry Package
\usepackage[top=1.0in,
bottom=1.0in,
right=1.0in,
left = 1.0in]{geometry}


%math
\usepackage{mathtools}


%Set up links and internal references
\usepackage[colorlinks=true,linkcolor=blue, urlcolor=blue]{hyperref} %hide the ugly red boxes
\usepackage[noabbrev]{cleveref} %don't abbreviate ``figure'', etc.
\usepackage{url} %for urls

%Remove natural indentation for itemize and enumerate environments
\usepackage{enumitem} %package to control itemize/enumerate behavior
\setlist[itemize]{leftmargin=*} %for itemize
\setlist[enumerate]{leftmargin=*} %for enumerate
%Swap out the itemize symbol for an N-dash rather than a bullet (does not require enumitem package)
\def\labelitemi{--}


%Set up fancy header and footer
\usepackage{fancyhdr}
\pagestyle{fancy}
%Fancy Header Content
\fancyhead[L]{ME EN 497R Syllabus}
\fancyhead[C]{Semester YYYY}
\fancyhead[R]{Student Name}
%Fancy Footer Content
\fancyfoot[L,C]{} %make bottom left and center empty
\fancyfoot[R]{\thepage}

%Center section headings
\usepackage{sectsty}
\sectionfont{\centering}
%Remove section numbering
\setcounter{secnumdepth}{0}




%%%%%%%%%%%%%%%%%%%%%%%%%%%%%%%%
%%%%%     END PREAMBLE     %%%%%
%%%%%%%%%%%%%%%%%%%%%%%%%%%%%%%%


%BEGIN Document Environment
\begin{document}


%--------------------------%
%----     OVERVIEW     ----%
%--------------------------%
\section{Overview}
\label{sec:overview}

%NOTE: This section has been written for the student's information.  Make sure they see it before you overwrite it with the overview for their semester.
ME EN 497R is a research course that may be applied toward the technical elective requirement for the Mechanical Engineering program (and occasionally other programs, e.g. Applied Physics).
The overall goal of the 497R course is to give undergraduates research experiences typically reserved for graduate students, albeit in an format accessible to the abilities and time constraints of an undergraduate student.
In the FLOW Lab, under the direction of Dr. Ning, we have somewhat formalized this course while also allowing a great deal of student-lead learning.
This syllabus document is the beginning of that process, wherein the undergraduate and graduate mentor work together to develop a customized syllabus for the undergraduate.
This document is a template to be filled out and submitted to Dr. Ning for final approval and an add code for ME EN 497R.





%-------------------------------%
%----     PREREQUISITES     ----%
%-------------------------------%
\section{Prerequisites}
\label{sec:prerequisites}

It is assumed that the student has completed the FLOW Lab Introductory Project, which includes a basic familiarity with the following concepts:

\begin{itemize}
	\item The Julia language, including using packages, functions, loops, doctstrings, comments, etc.
	\item LaTeX, including basic document creation, section headers, including figures, etc.
	\item Creating a github repository and using git for basic add, commit, pull, and push commands.	
\end{itemize}





%--------------------------%
%----     OUTCOMES     ----%
%--------------------------%
\section{Learning Outcomes}
\label{sec:learningoutcomes}


% --- Technical Outcomes
\subsection{Technical}
\label{ssec:technicaloutcomes}

\begin{enumerate}
	\item[1.] A basic understanding of the capabilities and weaknesses of, and practical experience using, the Vortex Lattice Method.
	\item[2.] A basic understanding of airfoil physics and practical experience analyzing airfoils and generating airfoil datasets.
	\item[3.] A basic understanding of optimization principles and practical experience applying optimization to a simple aerodynamic shape optimization.
\end{enumerate}


% --- Auxiliary Outcomes
\subsection{Auxiliary}
\label{ssec:auxiliaryoutcomes}

The following auxiliary learning outcomes are based on those tools and methods of communication that are \textit{universal} in the FLOW Lab.  Note that there are other auxiliary skills that may be gained depending on your approach to your chosen course design.

\begin{enumerate}
	\item Coding: The student, utilizing the Julia language, will learn the follwing:
	\begin{itemize}
		\item Docstring usage
		\item Good commenting practice
		\item Good code organization
		\item How to define, construct, and use parametric composite type structures
	\end{itemize}
	\item Version Control: Using git, the student will gain familiarity with the following concepts:
	\begin{itemize}
		\item Basic github repository organization and settings
		\item Submitting and responding to github Issues
		\item Opening and reviewing github Pull Requests
	\end{itemize}
	\item Technical Writing/Presenting: The student will gain introductory experience in the following areas:
	\begin{itemize}
		\item Reading and reviewing technical papers and creating bibliographies
		\item Writing basic versions of: technical papers, research proposals, and theses
		\item Presenting technical information in well designed figures and slides
		\item Presenting, both to subject matter experts as well as educated non-specialists
		\item Receiving and applying critical feedback
	\end{itemize}
\end{enumerate}





%------------------------------%
%----     GRADE RUBRIC     ----%
%------------------------------%
\section{Grade Rubric}
\label{sec:graderubric}

We have set up the semester schedule to resemble an approximation of a graduate student's timeline, but obviously condensed into a single semester time frame.
There are 8 major deliverables due throughout the semester.
\Cref{tab:rubric} shows the grade breakdown for those deliverables.
The following section provides detailed descriptions of the various deliverable contents and formats.

\begin{table}[h!]
	\caption{Proposed grade division for ME 497R.}
	\label{tab:rubric}
	\renewcommand{\arraystretch}{1.2}
	\vspace{1em}
	\begin{tabular}{c|l|l}
		\textbf{Points} & \textbf{Deliverable} & \textbf{Time to Complete}\\ 
		\hline
		5 & Intro Project Revision & 1 week\\
		10 & Paper 1 & 3 weeks \\
		15 & Prospectus + Review & 2 weeks + 1 week\\
		10 & Paper 2 & 3 weeks\\
		10 & Paper 3 & 3 weeks\\
		20 & Thesis + Defense& 2 weeks + 1 week \\
		70 & Time Log & updated every week		
	\end{tabular}
\end{table}





%--------------------------%
%----     SCHEDULE     ----%
%--------------------------%
\section{Deliverable Details}
\label{sec:deliverables}


% --- Orientation Assignment
%\vspace{1em}\hrule\vspace{1em}
\subsection{Project Orientation: Before you Begin}
\label{ssec:orientation}

You will need to complete the following before you begin.  Consider this part of the registration process.

\begin{enumerate}
	\item Complete this syllabus and receiving an add-code from Dr. Ning if you haven't already
	\item Schedule a weekly meeting time with your graduate student mentor.
	\item Set up your repository for developing, storing, and submitting your assignments for this semester. 
\end{enumerate}

\textit{Repository Setup:} You should have already used github to submit your Intro Project report and code. You will use that same method throughout this semester.
It will be very helpful to get your repository organized now.
You may consider the following:

\begin{enumerate}
	\item Rename your Intro Project repository on github something along the lines of ``497R Project'' or something similar.
	\item If you haven't already, add your graduate mentor as a collaborator so they can view your submissions throughout the semester
	\item Organize your repository in a way that makes sense for your course design. For example, you may consider moving all the intro project files to their own directory and adding a directory for memos/reports, a directory for code, etc.  Keep in mind that you need files in directories before git will recognize them.
	\item Initialize a README.md file in the root directory of your repository
	\item In the README, write a brief summary what your repository will be used for.  You can use this syllabus for ideas of what to include.  Make sure to include section headers and any other organizational elements to make it easy to read. (Think about how you could use this project as part of a portfolio.) You may also consider including a description of what your various directories include for easy reference later.
\end{enumerate}

\noindent TIP:  Remember that you can change any of this later.
More often than not, you'll update as you go and figure out better ways to organize, etc.



% --- Intro Project Revision Assignment
\vspace{1em}\hrule\vspace{1em}
\subsection{Intro Project Revision}
\label{ssec:ipr}

\textit{[date start - date end (Week 1)]}

\bigskip

Before enrolling in 497R, you should have finished and submitted your Intro Project to your assigned graduate student mentor for approval to move on to 497R.
If you haven't already, meet with your graduate mentor as soon as possible to receive feedback on your report and code.  Then take that feedback and revise your Introductory Project code and report.  Your revised code and report are your deliverables for this week.

As part of your report revisions deliverable, you should update your \LaTeX~document such that it can be used as a template for your written deliverables during this semester.
Your template should resemble the example document that can be found in the MEEN497R directory in the undergrad-onboarding repository.
Be sure to include a well commented preamble with any packages you might need (for things like math, figures, hyperlinks, etc.)
Note that this will likely not be a perfectly complete template, you will likely add to it later as your needs increase.  If you do this well, you should be able to use your template for your other course assignments (outside 497R), especially final reports.  Anecdotal evidence indicates that well formatted final reports (specifically in the 2-column peer-reviewed journal style format) such as those you can produce with \LaTeX~often lead to significant grade increases in most classes.

You can obviously use this syllabus document as a reference for your template, but make sure you understand what everything does rather than simply copy and pasting (there are some settings in this document purposely designed to cause problems if you simply copy and paste).



% --- Paper 1 
\vspace{1em}\hrule\vspace{1em}
\subsection{Paper 1: Vortex Lattice Method}
\label{ssec:p1}
\textit{[date start - date end (Weeks 2-4)]}

\bigskip

Complete the tasks and write a report based on the assignment definition found in the VortexLattice assignment document.

\bigskip

\noindent \textbf{Deliverables:}

\textit{Report:} Your written submission, for this and subsequent papers, should be written in the ``IMRaD'' format and include a bibliography with at least 3 references. 
You will also be graded on quality of your figures for this and all following assignments.
You may consider asking your graduate mentor for resources describing good figure design.
Be sure to use version control (see below) as you work on this assignment.

\textit{Version Control:}
For this, and every subsequent assignment, you will submit your deliverables using github.  Thus, for every assignment you should do the following: (note that this seems involved, but it's really very simple in practice)

\begin{enumerate}
	\item BEFORE you start coding, etc.~create a github issue with a comment describing the problem description and any other information you want to include.
	\item Make sure to add any assignments (assign yourself...) and labels (you may need to create custom labels) that make sense for your issue. You may consider using multiple issues in a single assignment, depending on how you've designed your course.
	\item Create a branch for what you're working on for this assignment.
	\item Throughout the assignment period, feel free to add comments and questions in your issue (this is a good way to write things down that you want to ask your graduate mentor in your weekly meetings).
	\item When you are ready to submit your assignment deliverables, create a pull request for your branch (this assumes you've been adding, committing, and pushing along the way as well). Request your graduate mentor as a reviewer.
	\item Upon approval of your pull request, merge with a comment that closes the issue for that assignment, and delete your branch (unless you need it for something else).
\end{enumerate}

\noindent REMEMBER: Do steps 1-3 before you start the coding and writing parts of this assignment. Also make sure you are using version control for your written papers in addition to your code.



% --- Prospectus
\vspace{1em}\hrule\vspace{1em}
\subsection{Prospectus}
\label{ssec:prospectus}
\textit{[date start - date end (Weeks 5-7)]}

\bigskip

As stated on the \href{https://www.wichita.edu/academics/fairmount_college_of_liberal_arts_and_sciences/english/deptenglish/WritingaResearchProspectus.php#:~:text=A%20prospectus%20is%20a%20formal,and%20will%20yield%20worthwhile%20results.}{Wichita State University Website}
A prospectus is a formal proposal of a research project developed to convince a reader (a professor or research committee, or later in life, a project coordinator, funding agency, or the like) that the research can be carried out and will yield worthwhile results. It should provide:

\begin{itemize}
	\item a working title for your project,
	\item a statement of your research question or issue,
	\item an overview of scholarship related to this topic or to the this author,
	\item a brief summary of your research methods and/or your theoretical approach.
\end{itemize}

\noindent \textbf{Deliverables:}

Your deliverables for this assignment are first, a written prospectus.
As you prepare your written prospectus, you may find the \href{https://www.me.byu.edu/00000177-4afb-d009-abf7-ebff31d50000/prospectus-outline-pdf-pdf}{ME department prospectus infographic} helpful.

The second deliverable is a review and revision of your prospectus.
You should complete your review-ready draft of your prospectus by the end of the second week of this assignment and submit it to your graduate mentor and another graduate student in the FLOW Lab (you will have to ask around in the FLOW Lab to see who has the time and experience to provide a review).
Using their feedback, revise your prospectus.

The format of your deliverable should be 3 documents:
\begin{itemize}
	\item Your review-ready draft, submitted by the 2-week mark for this assignment
	\item Your revised prospectus, submitted at the end of the assignment window
	\item A ``Diff'' document that highlights the changes you made between drafts.
\end{itemize}

To create your ``Diff'' document, you can either utilize multiple branches, or commits to save the versions of your prospectus.
You can then use a \LaTeX~diff tool (e.g. \href{https://texblog.org/2018/08/14/track-changes-with-latexdiff/}{latexdiff}) to highlight the changes you have made to your document.

\bigskip

TIP: Diff files and git merges are somewhat easier if you only put one sentence per line.
Remember a single newline will continue the paragraph in \LaTeX, while a double return will initiate a new paragraph.



% --- Paper 2 
\vspace{1em}\hrule\vspace{1em}
\subsection{Paper 2: Airfoil Families}
\label{ssec:p2}

\textit{[date start - date end (Weeks 8-10)]}

\bigskip

Complete the tasks and write a report based on the assignment definition found in the AirfoilFamilies assignment document.

\bigskip

\noindent \textbf{Deliverables:}

As part of this assignment, your code should include the following elements:
\begin{itemize}
	\item Typical code file organization structure
	\item At least one parametric composite type (that is used)
	\item Docstrings for every function
	\item Informative file headers
	\item Good commenting throughout code (you should have received feedback on this as part of your Paper 1 grading).
\end{itemize}


% --- Paper 3
\vspace{1em}\hrule\vspace{1em}
\subsection{Paper 3: Airframe Optimization}
\label{ssec:p3}

\textit{[date start - date end (Weeks 11-13)]}

\bigskip

Complete the tasks and write a report based on the assignment definition found in the AirframeOptimization assignment document.
	
\bigskip

% --- Final
\vspace{1em}\hrule\vspace{1em}
\subsection{Thesis and Defense}
\label{sec:finalreport}

\textit{[date start - date end (Weeks 14-16)]}

\bigskip

Your Thesis is a culmination of the papers produced thus far, revised according to mentor feedback, as well as any additional material necessary to produce a thesis that is both complete and well executed.

The Final Code Submission is similarly the final, cumulative state of the code produced as part of your assignments throughout the semester, revised according to mentor feedback, as well as any additional material necessary to understand and use the code produced.

In addition, you will need to schedule a Thesis Defense for some time during finals week.
You will schedule this with your graduate mentor, Dr. Ning, and one other graduate student in the FLOW Lab.
Your defense is really a presentation covering the contents in your Thesis, but in a condensed manner that highlights the main takeaways and your most significant contributions---think: what did \textit{you} do, and what were the most important things you found from your efforts.
You should expect to take roughly 15-20 minutes presenting in addition to a question and answer period.
You may also invite anyone else you would like to attend your presentation, but only the two graduate students and Dr. Ning will officially grade you.

Your Thesis should be written using the provided Thesis template, completed with the required frontmatter material.
Your Thesis and Final Code are due at the end of the day, on the last day of classes, and must be sent to the graduate students who will be attending your Defense as well as Dr. Ning, so that they have time to review your Thesis before your Defense.
You must complete your Defense presentation by the end of Finals week.


% --- Time Log
\vspace{1em}\hrule\vspace{1em}
\subsection{Time Log}
\label{sec:timelog}

\begin{quote}
	\textit{``The expectation for undergraduate courses is three hours of work per week per credit hour for the average student who is appropriately prepared; much more time may be required to achieve excellence.'' }
	
	--- BYU Policy and Procedures Catalog
\end{quote}


For 497R Credit in the FLOW Lab, we set our minimum time standard at 10 hours per week, which is only 1 hour more than the university standard for the average, appropriately prepared student.  
Each week, you will report the number of hours you spent that week, with the expectation that you will spend at least 10 hours.
At the end of the semester, you will be graded on your \textit{average} hours spent per week.
This allows to to have \textit{some} weeks with fewer than 10 hours, as long as you also have some weeks with more to make up for any deficiencies.
This time log makes up 50\% of your overall grade, so make sure you put in the required time.
Your report is honor based, so remember \href{https://www.churchofjesuschrist.org/study/scriptures/bofm/2-ne/9?lang=eng}{2 Nephi 9:34}.

It should be noted that spending the minimum average time will not guarantee excellent grades on your other assignments.  
You will not receive extra credit on your final grade for exceeding the minimum average time per week, but you will receive higher grades, and potentially extra credit (in the case of truly exceptional work) on your other assignments if you spend quality time on them.

\vspace{1em}\hrule\vspace{1em}




%---------------------------------%
%----     ACKNOWLEDGEMENT     ----%
%---------------------------------%
\section{Student/Mentor Acknowledgment}
\label{sec:acknowledgement}

I, [the student], have helped write, and have read, and I understand the above proposed syllabus and accept it as my desired 497R experience.

\vspace{2em}

\noindent \makebox[2.5in]{\hrulefill} \hspace {1.0in}\makebox[2.5in]{\hrulefill} \\
Student Signature \makebox[2.5in][r]{ Date} \\

\bigskip

I, [the mentor], have read the above proposed syllabus and accept responsibility for providing guidance, feedback, and grades as required by this course plan.

\vspace{2em}

\noindent \makebox[2.5in]{\hrulefill} \hspace {1.0in}\makebox[2.5in]{\hrulefill} \\
Mentor Signature \makebox[2.5in][r]{ Date} \\

\end{document}