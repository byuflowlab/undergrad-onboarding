\documentclass{article}% insert '[draft]' option to show overfull boxes 

%set up margins (gemetry package does more than just margins.)
\usepackage{geometry}
\geometry{
	letterpaper,
	left=1in,
	top=1in,
	bottom=1in}

%% Useful packages
%Nominal Packages
\usepackage{amsmath,xparse} %package for math stuff
\usepackage{esint} %package for cauchy principal value integral \fint
\NewDocumentCommand{\qfrac}{smm}{%
	\dfrac{\IfBooleanT{#1}{\vphantom{\big|}}#2}{\mathstrut #3}%
} % spaces out fractions

\usepackage{graphicx} %package for floats (figures)
\usepackage[colorlinks=true, allcolors=blue]{hyperref} %pacakge for hyperlinks (internal and external)
\usepackage{indentfirst} %package that indents first sentence of each paragraph.

%Advanced Packages
\usepackage{xfrac} %package that allows slanted fractions rather than stacked ones, use \xfrac{}{} instead of \frac{}{}
\usepackage{siunitx} %package for SI units (see ftp://ftp.dante.de/tex-archive/macros/latex/exptl/siunitx/siunitx.pdf)
\usepackage{gensymb} %package for some convenient things in both math and text mode (see http://ctan.math.illinois.edu/macros/latex/contrib/was/gensymb.pdf)
\usepackage{caption} %package for advanced captioning
\usepackage{subfigmat}% packages automating layout for subfigures
\usepackage{tcolorbox} %makes colored boxes

\usepackage{listings}
%\usepackage[usenames,dvipsnames]{xcolor}

\lstdefinelanguage{Julia}%
{morekeywords={abstract,break,case,catch,const,continue,do,else,elseif,%
		end,export,false,for,function,immutable,import,importall,if,in,%
		macro,module,otherwise,quote,return,switch,true,try,type,typealias,%
		using,while},%
	sensitive=true,%
	alsoother={\$},%
	morecomment=[l]\#,%
	morecomment=[n]{\#=}{=\#},%
	morestring=[s]{"}{"},%
	morestring=[m]{'}{'},%
}[keywords,comments,strings]%

\lstset{%
	language         = Julia,
	basicstyle       = \footnotesize,
	keywordstyle     = \bfseries\color{blue},
	stringstyle      = \color{magenta},
	commentstyle     = \color{gray},
	showstringspaces = false,
	breaklines=true,
	breakindent=0pt,
	tabsize=4,
}


\usepackage{fancyhdr}
\pagestyle{fancy}
\fancyhf{}
\rhead{Last Updated\\\today}
\chead{Julia Coding Activity 2:\\Control Flow}
\lhead{FLOW Lab\\LTRAD Program}
\rfoot{}
\lfoot{}


%Document Actually begins here.
\begin{document}
	
	The purpose of this assignment is to gain familiarity with basic control flow commands: if, elseif, else, for, while, etc. You can learn about these with the \href{https://en.wikibooks.org/wiki/Introducing_Julia/Controlling_the_flow}{Introducing Julia} page.
	
	\section*{If/Else}
	To practice if statements you'll create a camber function associated with the \href{https://en.wikipedia.org/wiki/NACA_airfoil}{NACA 4-Series Airfoils} 
	
	\bigskip
	
	\begin{tcolorbox}
		\textbf{
			NACA 4-series Camber Formula
		} 
		\textit{
			The NACA 4-series formula for airfoil camber along the chord is given by the following piecewise olynomial function, where $c$ is the value of maximum camber (as a percentage of the chord), p is the position of maximum camber (in units of chord/10), and $x$ is the x-position along the chord.
		}
		\begin{align*}
		\bar{z} = 
		\begin{cases}
		\qfrac{c(2cx-x^2)}{p^2} &\text{if  } x \leq p\\
		\qfrac{c(1-2c +2cx - x^2)}{(1-p)^2} &\text{if  } x > p
		\end{cases}
		\end{align*}
		
	\end{tcolorbox}
	
	\bigskip
	
	Using the NACA 4-series Camber Formula above, create a function that takes in three inputs: $x$, $p$, and $c$; and outputs the camber $\bar{z}$.
	
	To check your work, you should be able to recreate the following table for $c=0.02$ and $p=0.40$
	
	\renewcommand{\arraystretch}{1.2}
	\begin{tabular}{ l | l }
		$x$ & $\bar{z}$ \\
		\hline
		0.0 & 0.0\\
		0.1 & 0.00875\\
		0.2 & 0.015\\
		0.3 & 0.01875\\
		0.4 & 0.02\\
		0.5 & 0.019444\\
		0.6 & 0.01777\\
		0.7 & 0.015\\
		0.8 & 0.0111\\
		0.9 & 0.006111\\
		1.0 & 0.0\\
	\end{tabular}
	
	
	\section*{Loops}
	\begin{enumerate}
		\item Using for loops, print the tables from activities 1 and 2 using 3 lines of code each. (hint, call your function in the loop and use the iterator in your inputs)
		\item Do the same using while loops (you'll need 2 more lines of code to initialize and update the iterator).
	\end{enumerate}
	
\end{document}