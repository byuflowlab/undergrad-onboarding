%%%%%%%%%%%%%%%%%%%%%%%%%%%%
%%%%%     PREAMBLE     %%%%%
%%%%%%%%%%%%%%%%%%%%%%%%%%%%

\documentclass[12pt]{article}

%Define Margins using Geometry Package
\usepackage[top=1.0in,
bottom=1.0in,
right=1.0in,
left = 1.0in]{geometry}


%math
\usepackage{mathtools}


%Set up links and internal references
\usepackage[colorlinks=true,linkcolor=blue, urlcolor=blue]{hyperref} %hide the ugly red boxes
\usepackage[noabbrev]{cleveref} %don't abbreviate ``figure'', etc.
\usepackage{url} %for urls

%Remove natural indentation for itemize and enumerate environments
\usepackage{enumitem} %package to control itemize/enumerate behavior
\setlist[itemize]{leftmargin=*} %for itemize
\setlist[enumerate]{leftmargin=*} %for enumerate
%Swap out the itemize symbol for an N-dash rather than a bullet (does not require enumitem package)
\def\labelitemi{--}

\usepackage{multicol} %Allow for two columns in the middle of a paper. 


%Set up fancy header and footer
\usepackage{fancyhdr}
\pagestyle{fancy}
%Fancy Header Content
\fancyhead[L]{ME EN 497R}
\fancyhead[C]{}
\fancyhead[R]{Airframe Analysis Assignment}
%Fancy Footer Content
\fancyfoot[L,C]{} %make bottom left and center empty
\fancyfoot[R]{\thepage}

%Center section headings
\usepackage{sectsty}
\sectionfont{\centering}
%Remove section numbering
\setcounter{secnumdepth}{0}




%%%%%%%%%%%%%%%%%%%%%%%%%%%%%%%%
%%%%%     END PREAMBLE     %%%%%
%%%%%%%%%%%%%%%%%%%%%%%%%%%%%%%%


%BEGIN Document Environment
\begin{document}
	
%%%%%%%%%%%%%%%%%%%%%%%%%%%%%%%%
%%%%%      BACKGROUND      %%%%%
%%%%%%%%%%%%%%%%%%%%%%%%%%%%%%%%

\section{Background}

	The Vortex Lattice Method (VLM) is an invscid method for the low-fidelity analysis of lifting bodies (wings).
	For this assignment, you will be utilizing a Julia package called VortexLattice.jl to explore aspects of wing aerodynamic performance.
	
%%%%%%%%%%%%%%%%%%%%%%%%%%%%%%%%
%%%%%      ASSIGNMENT      %%%%%
%%%%%%%%%%%%%%%%%%%%%%%%%%%%%%%%
	
\section{Assignment}
Start a new branch on your repository. Name it something relevant to the project. Create github issues for at least steps 3-5. You should use these issues to write down questions you have about those steps of the assignment for reference in your weekly meeting with your graduate student mentor. Close each issue you create with a comment as you finish each step. 

\begin{enumerate}
	\item Read  through the chapter 4 in the ME EN 515 Text (linked below).  As you read and come across unfamiliar terms (see hint below), look them up and include them in an appendix in your report.  As part of the definitions, include images and equations to add clarity where applicable.
	\begin{itemize}
		\item[-] You may consider, rather than including an appendix in your papers, producing a dictionary of terms for yourself using the wiki feature on github.
	\end{itemize}
	\item Complete the \href{https://flow.byu.edu/VortexLattice.jl/stable/guide/}{Getting Started} and \href{https://flow.byu.edu/VortexLattice.jl/stable/examples/#Steady-State-Analysis-of-a-Wing}{3 Steady State Examples} in the \href{https://flow.byu.edu/VortexLattice.jl/stable/}{VortexLattice.jl documentation} Take notes on which functions you'll need for the rest of the assignment and how to use them (i.e. write some pseudo code). 
	\item Explore and discuss the effects of wing aspect ratio vs wing efficiency. 
	\item Explore the effects of tail volume ratio on the stability derivatives of an airframe. Be sure to discuss desirable signs for stability derivatives.
	\item Explore the effects on angle of attack on the lift coefficient. Discuss the limitations of the VLM and explain which of your results are wrong due to those limitations.
	\item Write a short write-up on your methods, results, and takeaways. You should include introduction and discussion material on what you learned in steps 1-5, giving special attention to the methods and results from steps 3-5. 
		\item Submit your code and paper (.tex and .pdf files) via a pull request for your assignment branch on github. In this assignment we expect a higher level of coding. We will specifically be look for the following: 
	\begin{itemize}
		\item Most of your code should be written in functions, with a small script to call your functions. Functions are faster and more modular that writing everything in scripts.  Every function should have a \href{https://docs.julialang.org/en/v1/manual/documentation/#Writing-Documentation}{docstring} explaining the purpose, inputs, and outputs. 
		\item All of your code should be well commented. This doesn't mean that every line must have a comment, but sufficient comments throughout are required. Comments should make the intent of the code explicit, as well as provide a road map. Any complicated line of code should be commented. 
	\end{itemize}
\end{enumerate}

\bigskip

\paragraph{Hint:} Here are some common terms that you may want to include in your appendix dictionary. You should also include other terms you come across that are unfamiliar.

\begin{multicols}{2}
	\begin{itemize}
		\item Coefficient of Drag (3D), $C_D$
		\item Coefficient of Lift (3D), $C_L$
		\item Coefficient of Moment (3D), $C_M$
		\item Wing Mean Aerodynamic Chord, $\overline{c}$
		\item Wing Span, $b$
		\item Wing Dihedral, $\phi$
		\item Wing Twist, $\theta$
		\item Wing Sweep, $\Lambda$
		\item Wing Aspect Ratio
		\item Tail Volume Ratios (horizontal and vertical)
		\item Inviscid Span Efficiency, $e$
		\item Airframe Stability Derivatives
	\end{itemize}
\end{multicols}



%%%%%%%%%%%%%%%%%%%%%%%%%%%%%%%%
%%%%%      RESOURCES       %%%%%
%%%%%%%%%%%%%%%%%%%%%%%%%%%%%%%%

\section{Useful Resources}

\begin{itemize}
	\item \href{https://byu.box.com/shared/static/ywfayozbj3sr2ot6b32u8nqk5brqvurt.pdf}{ME EN 515 Book (chapter 4 specifically)}
	\item \href{https://flow.byu.edu/VortexLattice.jl/stable/}{VortexLattice.jl Documentation}
	\item \href{https://letmegooglethat.com/?q=what+is+the+vertical+tail+volume+ratio+formula}{Google}
	\item \href{https://docs.github.com/en/communities/documenting-your-project-with-wikis/adding-or-editing-wiki-pages}{Adding wiki pages to your repository.}
\end{itemize}

%
%\section{Rubric}
%
%The rubric for this assignment is broken into two categories. 
%\Cref{tab:paperrubric} shows details for the paper submission rubric, and
%\cref{tab:coderubric} shows details for the code submission rubric.
%
%Note that you will be receiving detailed feedback on the quality of your figures for this assignment, but you will not be graded on figure quality until the next assignment.
%You should, however, include figures in this paper in order to receive feedback; otherwise you won't know what will be expected on subsequent assignments.
%
%\begin{table}[h!]
%	\caption{Approximate Rubric for Paper (subject to change)}
%	\label{tab:paperrubric}
%	\renewcommand{\arraystretch}{1.2}
%	\vspace{1em}
%	\begin{tabular}{r|p{5in}}
%		\textbf{Points} & \textbf{Item} \\ 
%		\hline
%		5 & Commented packages in document preamble \\
%		3 & Relevant title and author information \\
%		3 & Helpful Introduction \\
%		6 & Detailed Methodology \\
%		6 & Clear Presentation of Results \\
%		6 & Included Figures, with correct internal references \\
%		6 & Reasonable Discussion \\
%		\textbf{35} & \textbf{Total}
%	\end{tabular}
%\end{table}
%
%\begin{table}[h!]
%	\caption{Approximate Rubric for Code (subject to change)}
%	\label{tab:coderubric}
%	\renewcommand{\arraystretch}{1.2}
%	\vspace{1em}
%	\begin{tabular}{c|p{5in}}
%		\textbf{Points} & \textbf{Item} \\ 
%		\hline
%		5 & Create a new issue for this project on your git repository\\
%		5 & Create a new local branch for this project and commit regularly (not just right at the end). \\
%		5 & Push your branch to your remote origin and submit a pull request.\\
%		5 & Go through the review process with your graduate mentor and accept and merge the pull request. \\
%		5 & Add a Q\&A section to your git repository wiki where you write down questions to ask your graduate student mentor (or Google), then fill in the answers when you get them. \\
%		\textbf{25} & \textbf{Total}
%	\end{tabular}
%\end{table}
	
\end{document}