%%%%%%%%%%%%%%%%%%%%%%%%%%%%
%%%%%     PREAMBLE     %%%%%
%%%%%%%%%%%%%%%%%%%%%%%%%%%%

\documentclass[11pt,twocolumn]{article}

%Define Margins using Geometry Package
\usepackage[top=1.0in,
bottom=1.0in,
right=1.0in,
left = 1.0in]{geometry}


%math
\usepackage{mathtools}


%Set up links and internal references
\usepackage[colorlinks=true,linkcolor=blue, urlcolor=blue]{hyperref} %hide the ugly red boxes
\usepackage[noabbrev]{cleveref} %don't abbreviate ``figure'', etc.
\usepackage{url} %for urls

%Remove natural indentation for itemize and enumerate environments
\usepackage{enumitem} %package to control itemize/enumerate behavior
\setlist[itemize]{leftmargin=*} %for itemize
\setlist[enumerate]{leftmargin=*} %for enumerate
%Swap out the itemize symbol for an N-dash rather than a bullet (does not require enumitem package)
%\def\labelitemi{--}

\usepackage{multicol} %Allow for two columns in the middle of a paper. 


%Set up fancy header and footer
\usepackage{fancyhdr}
\pagestyle{fancy}
%Fancy Header Content
\fancyhead[L]{ME EN 497R}
\fancyhead[C]{}
\fancyhead[R]{Airframe Design Theory Exploration: Part I}
%Fancy Footer Content
\fancyfoot[L,C]{} %make bottom left and center empty
\fancyfoot[R]{\thepage}


% Customize formatting for section headers
\renewcommand{\thesection}{Problem \arabic{section}}
\renewcommand{\thesubsection}{\arabic{section}.\alph{subsection}}

%%%%%%%%%%%%%%%%%%%%%%%%%%%%%%%%
%%%%%     END PREAMBLE     %%%%%
%%%%%%%%%%%%%%%%%%%%%%%%%%%%%%%%


%BEGIN Document Environment
\begin{document}



%%%%%%%%%%%%%%%%%%%%%%%%%%%%%%%%%%%%%%%%%%%%%%%%%%%%%%%%%%%%%%%%%%%%%%
%                                                                    %
%                               Reading                              %
%                                                                    %
%%%%%%%%%%%%%%%%%%%%%%%%%%%%%%%%%%%%%%%%%%%%%%%%%%%%%%%%%%%%%%%%%%%%%%
\section*{Relevant Reading}
\label{sec:reading}

\begin{itemize}
	\item \href{http://flowlab.groups.et.byu.net/me415/flight.pdf}{Flight Vehicle Design Book}
	\begin{itemize}
		\item Chapter 2: Fundamentals (13 pages)
		\item Chapter 3: Drag (28 pages)
	\end{itemize}

\end{itemize}




%%%%%%%%%%%%%%%%%%%%%%%%%%%%%%%%%%%%%%%%%%%%%%%%%%%%%%%%%%%%%%%%%%%%%%
%                                                                    %
%                             Vocabulary                             %
%                                                                    %
%%%%%%%%%%%%%%%%%%%%%%%%%%%%%%%%%%%%%%%%%%%%%%%%%%%%%%%%%%%%%%%%%%%%%%

\section{Vocabulary}
\label{sec:vocab}

Explain the following terms; making sure to use sufficient detail, including any math or helpful figures.
In some cases, these terms are simple one sentence definitions, in others, you should include several paragraphs to explain them fully.

\paragraph{Wing Geometry Terms}
\begin{itemize}
	\item Wing Area
	\item Chord
	\begin{itemize}
		\item Mean Geometric Chord
		\item Mean Aerodynamic Chord
	\end{itemize}
	\item Taper Ratio
	\item Span
	\item Aspect Ratio
	\item Sweep
	\item Dihedral
	\item Twist
	\item Washout
\end{itemize}

\paragraph{Forces and Moments}
\begin{itemize}
	\item Lift
	\item Drag
	\begin{itemize}
		\item Induced Drag
		\item Parasitic Drag
		\begin{itemize}
			\item Skin Friction Drag
			\item Pressure Drag
		\end{itemize}
		\item Compressibility Drag
	\end{itemize}
	\item Pitching Moment
	\item Lift and Drag Polars
	\begin{itemize}
		\item Angle of Attack
		\item Zero Lift angle of attack
		\item Lift Curve Slope
		\item Stall
	\end{itemize}
\end{itemize}




%%%%%%%%%%%%%%%%%%%%%%%%%%%%%%%%%%%%%%%%%%%%%%%%%%%%%%%%%%%%%%%%%%%%%%
%                                                                    %
%                               Explore                              %
%                                                                    %
%%%%%%%%%%%%%%%%%%%%%%%%%%%%%%%%%%%%%%%%%%%%%%%%%%%%%%%%%%%%%%%%%%%%%%
\section{Exploration}
\label{sec:exploration}

Complete the following exploration.

\subsection{Prerequisites}
\label{ssec:prereqs}

\begin{enumerate}[label=\roman*.]
	\item \href{https://flow.byu.edu/VortexLattice.jl/stable/#Installation}{Install VortexLattice.jl} and complete the \href{https://flow.byu.edu/VortexLattice.jl/stable/guide/}{Getting Started Guide} as well as the \href{https://flow.byu.edu/VortexLattice.jl/stable/examples/#Steady-State-Analysis-of-a-Wing-and-Tail}{Steady State Wing and Tail Example}.
	\item Obtain, and become familiar with, the following tools auxiliary to VortexLattice.jl including:
	\begin{itemize}
		\item The airfoil analysis code
		\item The strip theory and far-field drag codes
	\end{itemize}
\end{enumerate}


\subsection{Forces, Moments, and Polars}

\begin{enumerate}[label=\roman*.]
	\item Create the following plots using the example wing from the previous problem:
	\begin{itemize}
		\item Lift vs Angle of Attack
		\item Induced Drag (near and far field) vs Angle of Attack
		\item Moment vs Angle of Attack
		\item Lift vs Drag
		\item Lift/Drag vs Angle of Attack
	\end{itemize}
	\item Identify the lift curve slope in your lift vs angle of attack plot, and compare it to the lift curve slope from thin airfoil theory (\(2\pi\)).
\end{enumerate}





\end{document}