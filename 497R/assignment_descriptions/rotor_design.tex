%%%%%%%%%%%%%%%%%%%%%%%%%%%%
%%%%%     PREAMBLE     %%%%%
%%%%%%%%%%%%%%%%%%%%%%%%%%%%

\documentclass[12pt]{article}

%Define Margins using Geometry Package
\usepackage[top=1.0in,
bottom=1.0in,
right=1.0in,
left = 1.0in]{geometry}


%math
\usepackage{mathtools}


%Set up links and internal references
\usepackage[colorlinks=true,linkcolor=blue, urlcolor=blue]{hyperref} %hide the ugly red boxes
\usepackage[noabbrev]{cleveref} %don't abbreviate ``figure'', etc.
\usepackage{url} %for urls

%Remove natural indentation for itemize and enumerate environments
\usepackage{enumitem} %package to control itemize/enumerate behavior
\setlist[itemize]{leftmargin=*} %for itemize
\setlist[enumerate]{leftmargin=*} %for enumerate
%Swap out the itemize symbol for an N-dash rather than a bullet (does not require enumitem package)
\def\labelitemi{--}

\usepackage{multicol} %Allow for two columns in the middle of a paper. 


%Set up fancy header and footer
\usepackage{fancyhdr}
\pagestyle{fancy}
%Fancy Header Content
\fancyhead[L]{ME EN 497R}
\fancyhead[C]{}
\fancyhead[R]{Rotor Design Assignment}
%Fancy Footer Content
\fancyfoot[L,C]{} %make bottom left and center empty
\fancyfoot[R]{\thepage}

%Center section headings
\usepackage{sectsty}
\sectionfont{\centering}
%Remove section numbering
\setcounter{secnumdepth}{0}




%%%%%%%%%%%%%%%%%%%%%%%%%%%%%%%%
%%%%%     END PREAMBLE     %%%%%
%%%%%%%%%%%%%%%%%%%%%%%%%%%%%%%%


%BEGIN Document Environment
\begin{document}
	
%%%%%%%%%%%%%%%%%%%%%%%%%%%%%%%%
%%%%%      BACKGROUND      %%%%%
%%%%%%%%%%%%%%%%%%%%%%%%%%%%%%%%

\section{Background}

Now that you've become comfortable with the BEM, it's time to explore the design space.
That means we want to better understand how changing different defining geometric and mechanical characteristics (called design variables) of the the wind turbine to see how it affects the outputs of our model. 
Finally, we will design a better rotor. 


%%%%%%%%%%%%%%%%%%%%%%%%%%%%%%%%
%%%%%      ASSIGNMENT      %%%%%
%%%%%%%%%%%%%%%%%%%%%%%%%%%%%%%%

\section{Assignment}
Start a new branch on your repository. Name it something relevant to the project. Create and issue for at least step 3. Close any issues you create with a comment as you finish each step. 

\begin{enumerate}
	\item Read chapter 1 of the \href{http://flowlab.groups.et.byu.net/mdobook.pdf}{Engineering Design Optimization}. 
	\item Using the model you analyzed in your previous assignment determine what the most effective design variables for this model are. Determine what conditions you should analyze. Determine a good objective function.
	\item Design a rotor following these criteria: 
	\begin{itemize}
		\item Increase your objective value.
		\item The root bending moment may no greater that 110\% of the original. 
		\item Propellers and helicopters may not have torque requirements larger than 110\% of the original value. 
	\end{itemize}
	\item[] (\textit{optional}) You may consider conducting a basic optimization using an out of the box optimizer such as \href{https://github.com/JuliaNLSolvers/Optim.jl}{optim.jl}. 
	\item Write a report (paper) on your methods, results, and takeaways. You should include introduction and discussion material on what you learned in steps 1-3, giving special attention to the methods and results from step 3. Your report should:
	\begin{itemize}
		\item Contain all the elements required for Papers 1 \& 2.
		\item Use the provided \href{https://github.com/byuflowlab/undergrad-onboarding/tree/497R/497R/latex_templates/paper3_wind}{Paper 3 \LaTeX~Template}.
		\item Include a bibliography with at least 3 citations in the text.
	\end{itemize}
	\item Submit your code and paper (.tex and .pdf files) via a pull request for your assignment branch on github. In this assignment we expect a higher level of coding. We will specifically be look for the following: 
	\begin{itemize}
		\item At this point you should be using scripts for run files.  That is, nearly all of your code will be contained in functions, and you should create run files that will automatically run analyses and post process your results (e.g. plot things).  As part of your submission, you should include a run file that, without any adjustment, your graduate student mentor should be able to run and get all the outputs for your paper (or at least specific sections of your paper).
		\item At this point you may also strongly consider implementing \href{https://docs.julialang.org/en/v1/manual/types/#Composite-Types}{parametric composite types} into your code if it would be helpful for your code organization. 
	\end{itemize}
\end{enumerate}


%%%%%%%%%%%%%%%%%%%%%%%%%%%%%%%%
%%%%%      RESOURCES       %%%%%
%%%%%%%%%%%%%%%%%%%%%%%%%%%%%%%%

\section{Useful Resources}
\begin{itemize}
   		\item \href{http://flowlab.groups.et.byu.net/mdobook.pdf}{Engineering Design Optimization : Chapter 1}
   		\item \href{https://www.google.com/}{Google}
\end{itemize}	
	
	
\end{document}