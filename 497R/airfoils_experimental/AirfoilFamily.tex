%%%%%%%%%%%%%%%%%%%%%%%%%%%%
%%%%%     PREAMBLE     %%%%%
%%%%%%%%%%%%%%%%%%%%%%%%%%%%

\documentclass[12pt]{article}

%Define Margins using Geometry Package
\usepackage[top=1.0in,
bottom=1.0in,
right=1.0in,
left = 1.0in]{geometry}


%math
\usepackage{mathtools}


%Set up links and internal references
\usepackage[colorlinks=true,linkcolor=blue, urlcolor=blue]{hyperref} %hide the ugly red boxes
\usepackage[noabbrev]{cleveref} %don't abbreviate ``figure'', etc.
\usepackage{url} %for urls

%Remove natural indentation for itemize and enumerate environments
\usepackage{enumitem} %package to control itemize/enumerate behavior
\setlist[itemize]{leftmargin=*} %for itemize
\setlist[enumerate]{leftmargin=*} %for enumerate
%Swap out the itemize symbol for an N-dash rather than a bullet (does not require enumitem package)
\def\labelitemi{--}


%Set up fancy header and footer
\usepackage{fancyhdr}
\pagestyle{fancy}
%Fancy Header Content
\fancyhead[L]{ME EN 497R}
\fancyhead[C]{}
\fancyhead[R]{Airfoil Family Assignment A}
%Fancy Footer Content
\fancyfoot[L,C]{} %make bottom left and center empty
\fancyfoot[R]{\thepage}

%Center section headings
\usepackage{sectsty}
\sectionfont{\centering}
%Remove section numbering
\setcounter{secnumdepth}{0}




%%%%%%%%%%%%%%%%%%%%%%%%%%%%%%%%
%%%%%     END PREAMBLE     %%%%%
%%%%%%%%%%%%%%%%%%%%%%%%%%%%%%%%


%BEGIN Document Environment
\begin{document}

\section{Background}

Airfoils are the cross sections of wings and rotors, so airfoil performance directly affects the performance of any lifting object.
You can learn more about the theory from the ME 515 Book and/or a google search.

For this assignment, you will be utilizing a code produced by students in the FLOWLab.
It is a Julia package called Xfoil.jl.
You will probably want to go through the examples in the documentation to get familiar with how to use the code.

%[todo: add more details to the background]

\section{Assignment}
Once you are familiar with using Xfoil.jl, complete the following, and write it up in your paper using the template provided by your graduate student mentor (1 Column Journal Template):

\begin{enumerate}
	\item Briefly explore the effect of airfoil thickness on airfoil lift and drag. 
	\item Briefly explore the effect of airfoil camber on airfoil lift, drag, and moment.
	\item Briefly explore the effect of airfoil angle of attack on airfoil lift, drag, and moment.
	\item Generate a surrogate model for a family of airfoils of your choice/creation.
\end{enumerate}


\section{Useful Resources}

\begin{itemize}
	\item \href{https://byu.box.com/shared/static/ywfayozbj3sr2ot6b32u8nqk5brqvurt.pdf}{ME EN 515 Book (chapter 2 specifically)}
	\item \href{https://flow.byu.edu/Xfoil.jl/stable/}{Xfoil.jl Documentation}
	\item \href{https://web.mit.edu/drela/Public/web/xfoil/}{Original XFoil Documentation}
\end{itemize}

\section{Rubric}

The rubric for this assignment is broken into two categories. 
\Cref{tab:paperrubric} shows details for the paper submission rubric, and
\cref{tab:coderubric} shows details for the code submission rubric.

\begin{table}[h!]
	\caption{Approximate Rubric for Paper (subject to change)}
	\label{tab:paperrubric}
	\renewcommand{\arraystretch}{1.2}
	\vspace{1em}
	\begin{tabular}{r|p{5in}}
		\textbf{Points} & \textbf{Item} \\ 
		\hline
		5 & Well written prose appropriate for technical communication. \\
		5 & Correct Paper format (using provided rubric) with title and author information as well as properly commented preamble. \\
		5 & Demonstrated understanding of exploration items. \\
		6 & Quality explanation of surrogate model development. \\
		6 & High Quality Figures, with correct internal references. \\
		5 & Notable improvement in overall quality of paper (did you apply the feedback you've been receiving thus far). \\
		3 & Correct Bibliography and citations. \\
		\textbf{35} & \textbf{Total}
	\end{tabular}
\end{table}

\begin{table}[h!]
	\caption{Approximate Rubric for Code (subject to change)}
	\label{tab:coderubric}
	\renewcommand{\arraystretch}{1.2}
	\vspace{1em}
	\begin{tabular}{c|p{5in}}
		\textbf{Points} & \textbf{Item} \\ 
		\hline
		5 & Correct Code Submission and Version Control usage\\
		5 & Descriptive Comments in code and commits \\
		5 & Correct Docstrings in code\\
		5 & Code uses functions and scripts to produce all the results and figures included in the paper without modification \\
		5 & Code runs from a single run file without bugs on the long-term support release of Julia\\
		\textbf{25} & \textbf{Total}
	\end{tabular}
\end{table}



\end{document}