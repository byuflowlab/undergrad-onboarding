%%%%%%%%%%%%%%%%%%%%%%%%%%%%
%%%%%     PREAMBLE     %%%%%
%%%%%%%%%%%%%%%%%%%%%%%%%%%%

\documentclass[12pt]{article}

%Define Margins using Geometry Package
\usepackage[top=1.0in,
bottom=1.0in,
right=1.0in,
left = 1.0in]{geometry}


%math
\usepackage{mathtools}


%Set up links and internal references
\usepackage[colorlinks=true,linkcolor=blue, urlcolor=blue]{hyperref} %hide the ugly red boxes
\usepackage[noabbrev]{cleveref} %don't abbreviate ``figure'', etc.
\usepackage{url} %for urls

%Remove natural indentation for itemize and enumerate environments
\usepackage{enumitem} %package to control itemize/enumerate behavior
\setlist[itemize]{leftmargin=*} %for itemize
\setlist[enumerate]{leftmargin=*} %for enumerate
%Swap out the itemize symbol for an N-dash rather than a bullet (does not require enumitem package)
\def\labelitemi{--}


%Set up fancy header and footer
\usepackage{fancyhdr}
\pagestyle{fancy}
%Fancy Header Content
\fancyhead[L]{ME EN 497R}
\fancyhead[C]{}
\fancyhead[R]{Vortex Lattice Method Assignment A}
%Fancy Footer Content
\fancyfoot[L,C]{} %make bottom left and center empty
\fancyfoot[R]{\thepage}

%Center section headings
\usepackage{sectsty}
\sectionfont{\centering}
%Remove section numbering
\setcounter{secnumdepth}{0}




%%%%%%%%%%%%%%%%%%%%%%%%%%%%%%%%
%%%%%     END PREAMBLE     %%%%%
%%%%%%%%%%%%%%%%%%%%%%%%%%%%%%%%


%BEGIN Document Environment
\begin{document}
	
\section{Background}

	The Vortex Lattice Method is an invscid method for the low-fidelity analysis of lifting bodies (wings).
	You can learn more about the theory from the ME 515 Book and/or a google search.
	
	For this assignment, you will be utilizing a code produced by students in the FLOWLab.
	It is a Julia package called VortexLattice.jl.
	You will probably want to go through the Getting Started and Examples: Steady State Analysis of a Wing and Tail portions of the documentation to get familiar with how to use the code.
	
%	[todo: add more details to the background]
	
\section{Assignment}
	Once you are familiar with using VortexLattice.jl, complete the following trade studies and write them up in a paper using the template provided by your graduate student mentor (AIAA Conference Template):
	
	\begin{enumerate}
		\item Explore and discuss the wing aspect ratio vs wing efficiency. 
		\item Explore the effects of tail volume ratio on the stability derivatives of an airframe. Be sure to discuss desirable signs for stability derivatives.
		\item Explore the effects on angle of attack on the lift coefficient. Discuss the limitations of the VLM and explain which of your results are wrong due to those limitations.
	\end{enumerate}





\section{Useful Resources}

[todo: add links here to text resources, code documentation, etc.]
\begin{itemize}
	\item \href{https://byu.box.com/shared/static/ywfayozbj3sr2ot6b32u8nqk5brqvurt.pdf}{ME EN 515 Book (chapter 4 specifically)}
	\item \href{https://flow.byu.edu/VortexLattice.jl/stable/}{VortexLattice.jl Documentation}
	\item \href{https://letmegooglethat.com/?q=what+is+the+vertical+tail+volume+ratio+formula}{Google}
	\item \href{https://docs.github.com/en/communities/documenting-your-project-with-wikis/adding-or-editing-wiki-pages}{Adding wiki pages to your repository.}
\end{itemize}


\section{Rubric}

The rubric for this assignment is broken into two categories. 
\Cref{tab:paperrubric} shows details for the paper submission rubric, and
\cref{tab:coderubric} shows details for the code submission rubric.

Note that you will be receiving detailed feedback on the quality of your figures for this assignment, but you will not be graded on figure quality until the next assignment.
You should, however, include figures in this paper in order to receive feedback; otherwise you won't know what will be expected on subsequent assignments.

\begin{table}[h!]
	\caption{Approximate Rubric for Paper (subject to change)}
	\label{tab:paperrubric}
	\renewcommand{\arraystretch}{1.2}
	\vspace{1em}
	\begin{tabular}{r|p{5in}}
		\textbf{Points} & \textbf{Item} \\ 
		\hline
		5 & Commented packages in document preamble \\
		3 & Relevant title and author information \\
		3 & Helpful Introduction \\
		6 & Detailed Methodology \\
		6 & Clear Presentation of Results \\
		6 & Included Figures, with correct internal references \\
		6 & Reasonable Discussion \\
		\textbf{35} & \textbf{Total}
	\end{tabular}
\end{table}

\begin{table}[h!]
	\caption{Approximate Rubric for Code (subject to change)}
	\label{tab:coderubric}
	\renewcommand{\arraystretch}{1.2}
	\vspace{1em}
	\begin{tabular}{c|p{5in}}
		\textbf{Points} & \textbf{Item} \\ 
		\hline
		5 & Create a new issue for this project on your git repository\\
		5 & Create a new local branch for this project and commit regularly (not just right at the end). \\
		5 & Push your branch to your remote origin and submit a pull request.\\
		5 & Go through the review process with your graduate mentor and accept and merge the pull request. \\
		5 & Add a Q\&A section to your git repository wiki where you write down questions to ask your graduate student mentor (or Google), then fill in the answers when you get them. \\
		\textbf{25} & \textbf{Total}
	\end{tabular}
\end{table}
	
\end{document}