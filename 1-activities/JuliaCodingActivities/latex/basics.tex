\documentclass{article}% insert '[draft]' option to show overfull boxes 

%set up margins (gemetry package does more than just margins.)
\usepackage{geometry}
 \geometry{
	letterpaper,
	left=1in,
	top=1in,
	bottom=1in}

%% Useful packages
%Nominal Packages
\usepackage{amsmath} %package for math stuff
\usepackage{esint} %package for cauchy principal value integral \fint

\usepackage{graphicx} %package for floats (figures)
\usepackage[colorlinks=true, allcolors=blue]{hyperref} %pacakge for hyperlinks (internal and external)
\usepackage{indentfirst} %package that indents first sentence of each paragraph.

%Advanced Packages
\usepackage{xfrac} %package that allows slanted fractions rather than stacked ones, use \xfrac{}{} instead of \frac{}{}
\usepackage{siunitx} %package for SI units (see ftp://ftp.dante.de/tex-archive/macros/latex/exptl/siunitx/siunitx.pdf)
\usepackage{gensymb} %package for some convenient things in both math and text mode (see http://ctan.math.illinois.edu/macros/latex/contrib/was/gensymb.pdf)
\usepackage{caption} %package for advanced captioning
\usepackage{subfigmat}% packages automating layout for subfigures
\usepackage{tcolorbox} %makes colored boxes

\usepackage{listings}
%\usepackage[usenames,dvipsnames]{xcolor}

\lstdefinelanguage{Julia}%
{morekeywords={abstract,break,case,catch,const,continue,do,else,elseif,%
		end,export,false,for,function,immutable,import,importall,if,in,%
		macro,module,otherwise,quote,return,switch,true,try,type,typealias,%
		using,while},%
	sensitive=true,%
	alsoother={\$},%
	morecomment=[l]\#,%
	morecomment=[n]{\#=}{=\#},%
	morestring=[s]{"}{"},%
	morestring=[m]{'}{'},%
}[keywords,comments,strings]%

\lstset{%
	language         = Julia,
	basicstyle       = \footnotesize,
	keywordstyle     = \bfseries\color{blue},
	stringstyle      = \color{magenta},
	commentstyle     = \color{gray},
	showstringspaces = false,
	breaklines=true,
}

\lstdefinestyle{customc}{
	belowcaptionskip=1\baselineskip,
	breaklines=true,
	frame=L,
	xleftmargin=\parindent,
	language=Julia,
	showstringspaces=false,
	basicstyle=\footnotesize\ttfamily,
	keywordstyle=\bfseries\color{green!40!black},
	commentstyle=\itshape\color{purple!40!black},
	identifierstyle=\color{blue},
	stringstyle=\color{gray},
}

\usepackage{fancyhdr}
\pagestyle{fancy}
\fancyhf{}
\rhead{Last Updated\\\today}
\chead{Activity 0:\\Julia Basics}
\lhead{FLOW Lab\\LTRAD Program}
\rfoot{}
\lfoot{}


%Document Actually begins here.
\begin{document}

\section*{The Basics}
The \href{https://learnxinyminutes.com/docs/julia/}{Learn X in Y Minutes} page for Julia has more than you'll need to get started.  Using the Julia REPL, learn how to do basic math such as addition, subtraction, multiplication, division, exponents, etc.  When you're comfortable with that, create a .jl file and write a script with some basic math things. Search the \href{https://docs.julialang.org/en/v1/index.html}{Julia Documentation} and learn how to print out strings and variables (you may find print, println, and display useful terms to search). Then learn how to run your julia script file either in the terminal, or in an IDE if you're using one.


\section*{Headers}
Although not necessary, you'll probably want to get in the habit of putting commented headers in your coding files with information such as who the author is (you), when it was written, what the file does, and any major updates after initial publication.

\begin{lstlisting}[frame=single]

# Example Header
# FLOW Lab LTRAD Program
# Author: Judd Mehr
# Date: 2 April 2019
# This header is an example of what you might want to put at the top of your Julia files.
\end{lstlisting}


\section*{Style Guide}
We follow the Julia language \href{https://docs.julialang.org/en/v1/manual/style-guide/index.html}{Style Guide}. Read through it. You won't understand a lot of it at this point, but keep referring back to make sure you're coding in the accepted style (this makes it much easier for people to read your code).


\end{document}